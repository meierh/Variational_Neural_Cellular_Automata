%%
% !TeX program = lualatex
%%

\documentclass[
ngerman,
accentcolor=9c,% Farbe für Hervorhebungen auf Basis der Deklarationen in den
type=intern,
marginpar=false
]{tudapub}

\usepackage[english, main=english]{babel}
\usepackage[autostyle]{csquotes}

%Formatierungen für Beispiele in diesem Dokument. Im Allgemeinen nicht notwendig!
\let\file\texttt
\let\code\texttt
\let\pck\textsf
\let\cls\textsf

\begin{document}


	\title{DGM Project Report 1}
	\author{Xiaoyan Xue, Vigelius, Xiaoyan Xue, Yunhao Sun}
	\date{\today}

	\maketitle
	\section{What we have worked on this week}
	\begin{itemize}
		\item Data exploration(All members)
		\item Research on approaches to balancing the label set(All members)
		\item Dataset information discussion(All members):
		\begin{itemize}
			\item The HAM10000 dataset, a large collection of multi-source dermatoscopic images of common pigmented skin lesions.  The final dataset consists of 10015 dermatoscopic images which can serve as a training set for academic machine learning purposes. Cases  include a representative collection of all important diagnostic categories in the realm of pigmented lesions.
		\end{itemize}
		\item Task distribution(All members):
		\begin{itemize}
			\item The whole project is divided into three parts for three members:
		\end{itemize}
		\begin{itemize}
			\item \textbf{Dataset Collection and Dataset Processing(YunHao Sun): }
			\begin{itemize}
				\item Dataset Selection(HAM10000)
				\item Data Augmentation:  Here are some data augmentation methods in our plan:
				\begin{itemize}
					\item Rotation: Rotate the image by a certain angle (e.g., 90 degrees, 180 degrees).
					\item Flip (Horizontal and Vertical): Flip the image horizontally and/or vertically.
					\item Zoom: Randomly zoom into the image.
					\item Brightness and Contrast Adjustment: Adjust the brightness and contrast of the image.
					\item Color Jittering: Introduce random variations in the color of the image.
					\item Gaussian Noise: Add random Gaussian noise to the image.
				\end{itemize}
				\item Data Cleaning: Check for and handle any missing or corrupted data in the dataset. Ensure that each image is associated with the correct label.
				\item Image Resizing: Resize images to a consistent resolution.
				\item Normalization: Normalize pixel values to a common scale, typically between 0 and 1.
				\item Class Imbalance: Check for class imbalances and use data augmentation methods to balance them.









			\end{itemize}
			\item \textbf{Model Selection, Training and Evaluation(Xiaoyan):}
			\begin{itemize}
				\item Model Selection: \textbf{Need more information and discussion for this specific task.} Here are some models we are considering about:
				\begin{itemize}
					\item Convolutional Neural Networks (CNNs)
					\item Residual Networks (ResNet)
					\item Inception Networks (GoogLeNet)
					\item DenseNet
				\end{itemize}
				\item Model Training: Implement neural network models and train the data.
				\item Model Evaluation:
				\begin{itemize}
					\item Accuracy
					\item Precision
					\item Recall
					\item F1 Score
					\item Confusion Matrix



				\end{itemize}
			\end{itemize}
			\textbf{        \item UI Design and Implemention(Manuel Vigelius):}
			\begin{itemize}
				\item Designing possible functionalities of the final application, some possible functionalities might be:
				\begin{itemize}
					\item Basic Functionalities: Sign up, login, personal information, password
					\item Supported Input: Support camera to take images on mobile devices, or upload image manually
					\item Document Exporting:  Export a report which includes the image, predicted result and related information or medical advice
					\item Archives: Saving past results with different labels
				\end{itemize}
				\item UI implementation: Plans to use Flutter for both mobile devices and PC.
			\end{itemize}
		\end{itemize}
	\end{itemize}


	\section{Results, findings and problems}
	\begin{itemize}
		\item Need more discussion for model selection
		\item Data balancing with data augmentation or giving different weights in model training process is not decided yet.
		\item A little bit hurry for three members to finish this project, not sure whether have enough time.
	\end{itemize}

	\section{Future work}
	\begin{enumerate}
		\item Discuss details and confusions above. Planned next meeting on 23.12.2023.
		\item Follow the task distributions:
		\begin{enumerate}
			\item The data processing part should be finished by 27.12.2023(1.5 weeks).
			\item The model part should be finished by 07.01.2024(1.5 weeks).
			\item The UI part should be finished by 17.01.2024(1.5 weeks).
		\end{enumerate}
	\end{enumerate}



\end{document}
